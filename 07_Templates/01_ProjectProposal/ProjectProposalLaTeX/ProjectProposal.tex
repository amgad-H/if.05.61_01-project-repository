\documentclass[12pt]{article}
\usepackage{geometry}                % See geometry.pdf to learn the layout options. There are lots.
\geometry{letterpaper}                   % ... or a4paper or a5paper or ... 
\usepackage{graphicx}
\usepackage{amssymb}
\usepackage{amsthm}
\usepackage{epstopdf}
\usepackage[utf8]{inputenc}
\usepackage[usenames,dvipsnames]{color}
\usepackage[table]{xcolor}
\usepackage{hyperref}
\DeclareGraphicsRule{.tif}{png}{.png}{`convert #1 `dirname #1`/`basename #1 .tif`.png}

\theoremstyle{definition}
\newtheorem{example}{Example}

\newtheorem{ourVersion}{ \linebreak}

\newenvironment{explanation}{%
   \setlength{\parindent}{0pt}
   \itshape
   \color{blue}
}{}

\newcommand{\projectname}{Schulplaner}
\newcommand{\productname}{Education Planner}
\newcommand{\projectleader}{A. Hammash}
\newcommand{\documentstatus}{In process}
%\newcommand{\documentstatus}{Submitted}
%\newcommand{\documentstatus}{Released}
\newcommand{\version}{V. 1.0}

\begin{document}
\begin{titlepage}
\begin{flushright}
\includegraphics[scale=.5]{htlleondinglogo.png}\\
\end{flushright}

\vspace{10em}

\begin{center}
{\Huge Project Proposal} \\[3em]
{\LARGE \productname} \\[3em]
\end{center}

\begin{flushleft}
\begin{tabular}{|l|l|}
\hline
Project Name & \projectname \\ \hline
Project Leader & \projectleader \\ \hline
Document state & \documentstatus \\ \hline
Version & \version \\ \hline
\end{tabular}
\end{flushleft}

\end{titlepage}
\section*{Revisions}
\begin{tabular}{|l|l|l|}
\hline
\cellcolor[gray]{0.5}\textcolor{white}{Date} & \cellcolor[gray]{0.5}\textcolor{white}{Author} & \cellcolor[gray]{0.5}\textcolor{white}{Change} \\ \hline
October 10, 2021&H. Hajredini/ A. Hammash/ D. Fetaj&First version \\ \hline
\end{tabular}
\pagebreak

\tableofcontents
\pagebreak

\section{Introduction}

\begin{ourVersion}
Our project focusses on making it easier for students to keep track of their due dates. 
We are planning on using android studio to make this app. 
By daily reminders, an overview of the next deadline-days and also a week and month overview of the schools timetable, 
we want to get a different school planer than the usual ones. 
\end{ourVersion}
\pagebreak

\section{Initial Situation}

\begin{ourVersion}
Students already have an overview over their timetable by using Webuntis and they also can work with calendars where they can enter the next deadline-days, homework or other important things.
But by working with two or more apps one could lose the overview very easily and it also could be very annoying to continually change the app for things that could be possible to use in just one app. \linebreak \linebreak
Verbesserung: \linebreak
Genauere Beschreibung der jetzigen Situation (was besser gemacht werden muss) \linebreak
Was fehlt in den schon bestehenden Apps? Was werden wir ergänzen? \linebreak
It can be used as a class- school planer (Ergänzung!!!) 
\end{ourVersion}
\pagebreak

\section{General Conditions and Constraints}

\begin{ourVersion}
The proposed system has to the deal with the following constraints: \linebreak

Technical Constraints: \linebreak

The app should sync with Webuntis and also a
backend is necessary to store common events (for the whole class) centrally.
Further on, Webuntis, common class events and personal events have to be merged in one view. \linebreak

Organisational Constraints: \linebreak
Userdata (which userdata?) has to be encrypted. \linebreak

How would this app be used by students?? (desktop version?)

\end{ourVersion}

\pagebreak

\section{Project Objectives and System Concepts}

\begin{ourVersion}
The project’s use case will usually be like this: \linebreak
By opening the app the students can see first what they have to do in the next time by a pop up which covers the whole screen and
the Webuntis-Sync helps them knowing when the next lessons are canceled.
They also can enter their Homework to its respective subject and schedule.
By this, they can easily see if they have homework to do or not, and when a subject is chosen a more informative window will pop up showing exactly what the homework is.
Students also have a clear overview of the next weeks or months. \linebreak
Our plan is to be able to work with all these features in our app, so students can plan their whole week or month more easily and by including the timetable, they also would have a much better overview over the canceled hours or the coming school events.
Homeworks and Tests shall be visually emphasized to remind the user of upcoming dead-lines. \linebreak
Bilder als Beispiele!!!
Wie wird auf die Server zugegriffen??

\end{ourVersion}

\pagebreak
\section{Opportunities and Risks}

 \begin{ourVersion}
The project’s opportunities are: \linebreak
Wieso wollen wir diese App entwickeln? \linebreak
Students won’t overlook homework and our app could reduce stress since the dates won’t be scattered around in various apps \linebreak	

Risks to consider: \linebreak
Was wären probleme bei der Entwicklung?\linebreak
Student may forget to enter their homework and also if class is split up, homework could be different.
	
 \end{ourVersion}


\pagebreak
\section{Planning}

\begin{ourVersion}
We plan on having the app show a synced-up week view from Webuntis and have it be interactable 
so students can already write their homework on it, that will be our first mile-stone. 
The roles of the team members will be that same and we’ll divide the workload be-tween us. 
The app will be made in Android studio and the files may have to be saved lo-cally.
\end{ourVersion}

\end{document}  